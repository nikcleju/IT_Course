\documentclass[12pt,]{scrartcl}
\usepackage{lmodern}
\usepackage{amssymb,amsmath}
\usepackage{ifxetex,ifluatex}
\usepackage{fixltx2e} % provides \textsubscript
\ifnum 0\ifxetex 1\fi\ifluatex 1\fi=0 % if pdftex
  \usepackage[T1]{fontenc}
  \usepackage[utf8]{inputenc}
\else % if luatex or xelatex
  \ifxetex
    \usepackage{mathspec}
  \else
    \usepackage{fontspec}
  \fi
  \defaultfontfeatures{Ligatures=TeX,Scale=MatchLowercase}
\fi
% use upquote if available, for straight quotes in verbatim environments
\IfFileExists{upquote.sty}{\usepackage{upquote}}{}
% use microtype if available
\IfFileExists{microtype.sty}{%
\usepackage[]{microtype}
\UseMicrotypeSet[protrusion]{basicmath} % disable protrusion for tt fonts
}{}
\PassOptionsToPackage{hyphens}{url} % url is loaded by hyperref
\usepackage[unicode=true]{hyperref}
\hypersetup{
            pdftitle={Introduction},
            pdfborder={0 0 0},
            breaklinks=true}
\urlstyle{same}  % don't use monospace font for urls
\IfFileExists{parskip.sty}{%
\usepackage{parskip}
}{% else
\setlength{\parindent}{0pt}
\setlength{\parskip}{6pt plus 2pt minus 1pt}
}
\setlength{\emergencystretch}{3em}  % prevent overfull lines
\providecommand{\tightlist}{%
  \setlength{\itemsep}{0pt}\setlength{\parskip}{0pt}}
\setcounter{secnumdepth}{0}
% Redefines (sub)paragraphs to behave more like sections
\ifx\paragraph\undefined\else
\let\oldparagraph\paragraph
\renewcommand{\paragraph}[1]{\oldparagraph{#1}\mbox{}}
\fi
\ifx\subparagraph\undefined\else
\let\oldsubparagraph\subparagraph
\renewcommand{\subparagraph}[1]{\oldsubparagraph{#1}\mbox{}}
\fi

% set default figure placement to htbp
\makeatletter
\def\fps@figure{htbp}
\makeatother


\title{Introduction}
\providecommand{\subtitle}[1]{}
\subtitle{Information Theory Lab 1}
\date{}

\begin{document}
\maketitle

\section{Objective}\label{objective}

Getting familiar with the IDE and workflow for the Information Theory
laboratory.

\section{Theoretical notions}\label{theoretical-notions}

None today :)

\section{Practical issues}\label{practical-issues}

The laboratory applications will be mostly done in C.

We will use a general purpose C IDE (Code Blocks or Dev C++).

\section{C details}\label{c-details}

A C program may receive input arguments from the command line. For this,
the header of the \texttt{main()} function must be defined as follows:

\begin{verbatim}
int main(int argc, char* argv[])
{
    ...
}
\end{verbatim}

\subsection{Explanations:}\label{explanations}

\begin{itemize}
\tightlist
\item
  \texttt{argc} (``\textbf{arg}ument \textbf{c}ount'') holds the number
  of input arguments
\item
  \texttt{argv} (``\textbf{arg}ument \textbf{v}alues'') is an
  \emph{array of strings} which holds all the input arguments (separated
  by spaces)
\item
  the first argument \texttt{argv{[}0{]}} is always the full name of the
  executable
\end{itemize}

\subsection{Example:}\label{example}

\texttt{test.exe\ file1\ file2\ dostuff}

\begin{itemize}
\tightlist
\item
  \texttt{argc} is 4
\item
  \texttt{argv} is
  \texttt{\{"test.exe",\ "file1",\ "file2",\ "dostuff"\}}
\end{itemize}

\section{Exercises}\label{exercises}

\begin{enumerate}
\def\labelenumi{\arabic{enumi}.}
\tightlist
\item
  Write a C program that displays all its input arguments, one argument
  on a separate line.
\end{enumerate}

\section{Final questions}\label{final-questions}

\begin{enumerate}
\def\labelenumi{\arabic{enumi}.}
\tightlist
\item
  TBD
\item
  TBD
\end{enumerate}

\end{document}
